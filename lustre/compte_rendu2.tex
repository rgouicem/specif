\documentclass[a4paper, 11pt]{article}
 
\usepackage[utf8]{inputenc}
\usepackage{graphicx}
\usepackage[frenchb]{babel}
\usepackage{tikz}
\usetikzlibrary{arrows,automata}
\usepackage{makecell}
\usepackage{amsthm}

\newtheorem{propriete}{Propriété}

\begin{document}
 
\title{SPECIF\\Compte-rendu de TME}
\author{Redha Gouicem}
\date{09/03/2015}
 
\maketitle

\section{Feux de voitures}
\paragraph{Question 2}
Les propriétés suivantes ont été vérifiées sur notre modélisation des feux de voiture étendus grâce à \texttt{xlesar} (projet \texttt{feux\_etendue.lesar} dans l'archive):
\begin{propriete}
 Veilleuses, codes et phares sont exclusifs.
\end{propriete}
\begin{propriete}
  On ne peut être en anti-brouillard que si l'on est en codes.
\end{propriete}
\begin{propriete}
  On ne peut être en longue-portée que si l'on est en phares.
\end{propriete}

\section{\'Equivalence de systèmes}
\paragraph{Question 3}
On teste l'équivalence de ces deux noeuds \texttt{lustre}.
\begin{verbatim}
node n1 (a, b: bool) returns (s1, s2: bool);
let
  s1 = 0->a and pre b;
  s2 = 1->not pre s2;
tel

node n2 (a, b: bool) returns (s1, s2: bool);
let
  s1 = 0->not(a xor pre b);
  s2 = 1->not pre s2;
tel
\end{verbatim}
La séquence suivante, générée par \texttt{xlesar} montre que cette propriété est fausse :

\begin{verbatim}
--- TRANSITION 1 ---
not b
--- TRANSITION 2 ---
not a
\end{verbatim}

\section{Tramway}
\paragraph{Question 4}
On va d'abord expliciter les liens entre les différents acteurs du système (signaux échangés), à savoir le contrôleur, la porte, le tramway et l'usager.\newline
\begin{center}
\begin{tikzpicture}[->,>=stealth',shorten >=1pt,auto,node distance=2.8cm,
                    semithick]
  \tikzstyle{every state}=[fill=white,text=black]

  \node[state] (0) {porte};
  \node[state] (1) [right of=0] {controleur};
  \node[state] (2) [right of=1] {tramway};
  \node[state] (3) [below of=1] {usager};
  \path (0) edge [bend right] node {a} (1)
  (2) edge [bend right] node {b} (1)
  (3) edge node {c} (1)
  (1) edge [bend right] node {a'} (0)
  (1) edge [bend right] node {b'} (2);
\end{tikzpicture}
\end{center}

Avec :
\begin{itemize}
  \item a = porte\_ouverte
  \item b = en\_station, attention\_depart
  \item c = demande\_porte
  \item a' = ouvrir\_porte, fermer\_porte
  \item b' = porte\_ok
\end{itemize}

\paragraph{Question 5}
On souhaite désormais vérifier la propriété suivante :
\begin{propriete}
  On ne peut pas rouler avec la porte ouverte.
\end{propriete}
Avec le premier modèle (fichier \texttt{tramway.lus}), il est impossible de vérifier cette propriété puisque les signaux du tramway (\texttt{en\_station} notamment) sont des entrées du système. Rien ne nous empêche de démarrer (mettre \texttt{en\_station} à \texttt{false}) malgré que le signal \texttt{porte\_ok} soit \texttt{false}. La seconde version (\texttt{tramway2.lus}) tente de résoudre se problème en modélisant également le tramway. Cependant, cette version n'est pas fonctionnelle à ce jour.


\end{document}
